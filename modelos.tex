
\section{Definiciones}
\begin{enumerate}
\item {\bf Aserci\'on descriptiva}: cosas que son o presumimos verdaderas en el mundo. Taxonomias: 
  \begin{enumerate}
    \item {\bf Propiedades del Dominio}: son las aserciones descriptivas que son propiedades fisicas. 
    \item {\bf Hip\'otesis del Dominio}: propiedades fisicas que pueden ser cambiadas. 
  \end{enumerate}
  
\item {\bf Aserciones prescriptivas}: cosas que esperamos que sean verdaderas. Dos tipos: 
  \begin{enumerate}
    \item {\bf Objetivos} (multi-agentes): acerca de fenomenos en el mundo. [Cosas que esperamos sean ciertas en el mundo una vez afectado por el sistema. ]. [asercion prescriptiva que el sistema debera satisfacer a traves de la cooperacion de sus agentes. ]
    \item {\bf Requerimientos} (objetivo uni-agente): acerca de fenomenos en la interfaz. [Cosas que debemos hacer ciertas en la interfaz.]Taxonomias: 
      \begin{enumerate}
	  \item {\bf Expectativas}: responsabilidad de un agente externo
	  \item {\bf Requerimientos}: responsabilidad de nuestro sistema
      \end{enumerate}
  \end{enumerate}
  Los objetivos con fenomenos globales, los requerimientos son objetivos uniagentes concretos.

\item {\bf Operaci\'on}: es una funci\'on que toma un estado del mundo y devuelve otro estado, en el que solo cambiaron las variables controladas por la maquina. Son introducidas por un objetivo uni-agente (requerimiento) del modelo de objetivos. Los requerimientos (2.2.2) inducen operaciones del software, y las expectativas (2.2.1) inducen operaciones de agentes. Consta de: 
  \begin{itemize}
    \item Operaci\'on: nombre
    \item Responsable: actor responsable de la operaci\'on
    \item Usuarios
    \item Definici\'on: explicaci\'on de la operaci\'on
    \item Entrada
    \item Salida
    \item Pre-condicion: condicion necesaria para que la operacion pueda realizarse.
    \item Post-condicion: condicion garantizada despues de la operacion. 
    \item Trigger: condicion que si se da, la operacion ocurre. 
  \end{itemize}
  
\item {\bf Objeto conceptual}: denota una entidad o concepto del dominio del problema. Puede ser: objeto pasivo, activo, personas, estructuras, etc. 

\item {\bf Clase conceptual}: denota un conjunto de objetos conceptuales que comparten caracteristicas comunes. 

\item {\bf Comportamiento} es el conjunto de respuestas, reacciones o movimientos hechos por un organismo en cualquier situaci\'on. 

\end{enumerate}

\section{Modelos}

\subsection{Modeo de Jackson}

  Estructura el mundo, la maquina y la interfaz. Aserciones descriptivas y prescriptivas. 

\subsection{Modelo de agentes}

  Estructura para los fenomenos del mundo.

  \subsubsection{Diagrama de contexto}

  \begin{itemize}
    \item[+] \underline{Nodos}: agentes, m\'aquina. 
    \item[+] \underline{Aristas}: fen\'omenos
  \end{itemize}

  \ponerGrafico{images/contexto.pdf}{Diagrama de contexto}{0.6}{contexto}
  
\subsection{Modelo de objetivos}

  Estructura para las aserciones del mundo. Deberan estar dadas en funcion de fenomenos en la interfaz de agentes, y deben ser declarativas, no operacionales (describir el objetivo y no el como). 

  \subsubsection{Diagrama de objetivos}
  \begin{itemize}
    \item[+] \underline{Nodos}: Objetivos duros y blandos, agentes, presunci\'on del dominio
    \item[+] \underline{Aristas}: ``Contribuye a''
  \end{itemize}

  \ponerGrafico{images/objetivos.pdf}{Diagrama de objetivos}{0.6}{objetivos}

\subsection{Modelo de operaciones}

  Sirve para estructura las operaciones y poder validarlas. 

  \subsubsection{Diagrama de casos de uso}
  
  Describen bajo la forma de acciones y reacciones las operaciones provistas por una m\'aquina desde el punto de vista del usuario, pero solo interesan las interacciones m\'aquina-agente. 
  
  Delimita el alcance del software a construir. 
  
  \ponerGrafico{images/casos_de_uso.pdf}{Diagrama de Casos de Uso}{0.6}{casos_de_uso}
  
  \begin{itemize}
    \item[+] \underline{Nodos}: Actor, M\'aquina, Caso de uso. 
    \item[+] \underline{Aristas}: Participa en, Herencia, Inclusi\'on, Extend. 
  \end{itemize}
  
  Cada caso de uso involucra participaci\'on de actores, y esta tiene que ser explicada en el caso de uso detallado. 
  
  \subsubsection*{Caso de uso detallado} 
  
  Especifica una o varias secuencias de acci\'on que el sistema puede llevar a cabo interactuando con sus actores. 
  
  El nombre se expresa con un verbo en gerundio. 
  
  \begin{casodeuso}
    \cutitle{CU E}
    \cuactors{Actores que participan}
    \cupre{Precondicion del caso de uso}
    \cupost{Postcondicion}
    \cucourse{
    1. ... & \\
    2. EXTIENDE Caso de uso CU D & \\
    3. FIN CU & \\
    }
  \end{casodeuso}

  \begin{casodeuso}
    \cutitle{CU C}
    \cuactors{Actores que participan}
    \cupre{Precondicion del caso de uso}
    \cupost{Postcondicion}
    \cucourse{
    1. ... & \\
    2. USA Caso de uso CU B & \\
    3. FIN CU & \\
    }
  \end{casodeuso}
  
  Durante la ejecuci\'on de un caso de uso suelen aparecer errores o excepciones, lo que produce un desv\'io del curso normal al curso alternativo. 
  
  \subsubsection*{Sem\'antica de las relaciones}
  
  \begin{enumerate}
   \item[{\bf Extensi\'on:}] Decir que el caso de uso B extiende al caso de uso A, significa que hay instancias de A que incluir\'an, a veces, el comportamiento del caso de uso B. 
   
   Puede que hayan escenarios descritos por el caso de uso B que no aparecen en escenarios denotados por A. Ejemplo,
   
   \ponerGrafico{images/caso_de_uso_extend.pdf}{}{0.6}{casos_de_uso_extends}
   
   \item[{\bf Inclusi\'on:}] Decir que el caso de uso A usa (o incluye) al caso de uso B, significa que toda instancia de A incluye al comportamiento descrito por el caso de uso B. 
   
   Puede que hayan escenarios descritos por el caso de uso B que no aparecen en escenarios denotados por A. Ejemplo,
   
   \ponerGrafico{images/caso_de_uso_includes.pdf}{}{0.6}{casos_de_uso_extends}
  \end{enumerate}

 
\subsection{Modelo conceptual}

  Estructura la definicion de los conceptos (sustativos asociados al dominio de problema) mas relevantes. 

  \subsubsection{Diagrama de clases}
  \begin{itemize}
    \item[+] \underline{Nodos}: rect\'angulos con el nombre de la clase, y debajo sus atributos.
    \item[+] \underline{Aristas}: relaciones entre clases, con la cantidad y el nombre de la relaci\'on. 
  \end{itemize}

  Se usa {\bf OCL} para establecer las restricciones entre las relaciones.

\subsection{Modelo de comportamientos}

  Describen comportamientos de los agentes. Se separan en dos: basados en estados, y basados en interacciones. 

  \subsubsection{Diagrama de actividad}


